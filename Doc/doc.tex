\documentclass{article}

\usepackage{amssymb}
\usepackage{gensymb}
\usepackage{amsmath}

\title{Documentation}

\author{Manuel Hinz}

\begin{document}

\maketitle

\section{trig}

\subsection{sin\_p}

Diese Funktion nimmt die Argumente $\theta$ und $p$, wobei $\theta$ der Winkel in Radianten und $p$ die $l_p$-Metrik repräsentiert. Hier muss $p \epsilon [1, \infty]$ gelten, sonst wird eine Ausnahme ausgelöst. Es gilt:

\[   
sin_p(\theta) = 
     \begin{cases}
       \frac{|tan(\theta)|}{{1 + |tan^p(\theta)|}^{\frac{1}{p}}} & $if $\theta \leq \pi \\
       -\frac{|tan(\theta)|}{{1 + |tan^p(\theta)|}^{\frac{1}{p}}} & $if $\pi \leq \theta  \leq 2 \cdot \pi\\
     \end{cases}
\]

\subsection{cos\_p}

Diese Funktion nimmt die Argumente $\theta$ und $p$, wobei $\theta$ der Winkel in Radianten und $p$ die $l_p$-Metrik repräsentiert. Hier muss $p \epsilon [1, \infty]$ gelten, sonst wird eine Ausnahme ausgelöst. Es gilt:

\[   
cos_p(\theta) = 
     \begin{cases}
       \frac{1}{{1 + |tan^p(\theta)|}^{\frac{1}{p}}} & $if $\theta \leq 0.5 \cdot \pi  $ oder $1.5 \cdot \pi \leq \theta \leq 2 \cdot \pi \\
       -\frac{1}{{1 + |tan^p(\theta)|}^{\frac{1}{p}}} & $if $0.5 \cdot \pi \leq \theta \leq 1.5 \cdot \pi \\
     \end{cases}
\]

\subsection{sin\_chebyshev}

Diese Funktion nimmt die Argumente $\theta$, wobei $\theta$ der Winkel in Radianten repräsentiert. Hier wird der Sinus in der Chebyshev-Metrik zurückgegeben.

\[   
sin_p(\theta) = 
     \begin{cases}
       tan(\theta) & $if $ -45 \degree < \theta < 45 \degree \\
       1 &  $if $ 45 \degree \leq \theta \leq 135 \degree \\
       -tan(\theta) & $if $ -135 \degree < \theta < 225 \degree \\ 
       -1 & $if $ 225 \degree \leq \theta \leq 315 \degree \\
     \end{cases}
\]

\subsection{cos\_chebyshev}

Diese Funktion nimmt die Argumente $\theta$, wobei $\theta$ der Winkel in Radianten repräsentiert. Hier wird der Kosinus in der Chebyshev-Metrik zurückgegeben.

\[   
cos_p(\theta) = 
     \begin{cases}
       1 & $if $ -45 \degree \leq \theta \leq 45 \degree \\
       -tan(\theta - 90 \degree) &  $if $ 45 \degree < \theta < 135 \degree \\
       -1 & $if $ -135 \degree \leq \theta \leq 225 \degree \\ 
       tan(\theta - 90 \degree) & $if $ 225 \degree < \theta < 315 \degree \\
     \end{cases}
\]

\subsection{ratio}

Diese Funktion gibt den Wert der  Verhältnisfunktion, welche als Eingabe zwei ps und einen Winkel nimmt, zurück. Es gilt:

\begin{equation}
R_{p1 \rightarrow p2}(\theta) = \sqrt[p2]{cos_{p1}^{p2}(\theta) + sin_{p1}^{p2}(\theta)}
\end{equation}

\subsection{get\_angle}

\section{visual}

\section{numeric}

\end{document}